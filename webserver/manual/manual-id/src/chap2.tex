%% This is an example first chapter.  You should put chapter/appendix that you
%% write into a separate file, and add a line \include{yourfilename} to
%% main.tex, where `yourfilename.tex' is the name of the chapter/appendix file.
%% You can process specific files by typing their names in at the 
%% \files=
%% prompt when you run the file main.tex through LaTeX.
\chapter{Rumusan Masalah}
% Rumusan masalah mencakup :
%  Penjelasan mengenai masalah yang menjadi topik penelitian
%  Cakupan dan batasan penelitian
%  Esensi penelitian yang dilakukan

\section{Penjelasan masalah}
Masalah-masalah yang akan dipecahkan dalam penelitian ini meliputi:

\begin{enumerate} [topsep=0mm]
\itemsep0mm
\item bagaimana mengukur kesamaan dalam bentuk kernel dari dua data dalam ruang kimia, genomik atau farmakologi?
\item bagaimana menggabungkan informasi dari tiga ruang: kimia, genomik dan farmakologi?
\item bagaimana merumuskan metode prediksi berbasis kernel untuk estimasi interaksi senyawa-protein yang akurat, 
(dengan akurasi lebih besar dari 95\%)?
\item bagaimana memilah dan mengunduh secara otomatis informasi dari banyak database yang relevan?
\item bagaimana desain antar-muka yang user-friendly untuk aplikasi-web yang melayani estimasi interaksi senyawa-protein?
\item bagaimana desain web-server sehingga memiliki tingkat utility, maintainability dan scalability yang tinggi?
\end{enumerate}

\section{Cakupan dan batasan penelitian}
Cakupan dan batasan dari penelitian ini adalah sebagai berikut:

\begin{enumerate} [topsep=0mm]
\itemsep0mm
\item metode prediksi berbasis kernel
\item menggunakan kernel dari tiga ruang: kimia, genomik dan farmakologi
\item web-server menerima input berikut: 
	\begin{itemize} [topsep=0mm]
	\itemsep0mm
	\item dari aspek obat: nama tanaman atau nama senyawa
	\item dari aspek penyakit: nama penyakit atau nama protein
	\end{itemize}
\item informasi untuk prediksi berasal dari database publik yang tersedia
\item prediksi mencakup senyawa organik dari tanaman herbal dan senyawa sintetik dari obat sintetik
\end{enumerate}

\section{Esensi penelitian}
Esensi dari penelitian ini meliputi:

\begin{enumerate} [topsep=0mm]
\itemsep0mm
\item
Prediksi yang akurat terhadap interaksi senyawa bioaktif dengan protein merupakan informasi yang krusial untuk pengembangan obat dan suplemen makanan, baik herbal maupun sintetis.
Dengan informasi ini sebagai penapis in-silico, uji klinis yang mahal dan lama hanya dilakukan pada kandidat senyawa yang berpotensi tinggi.

\item
Web-server publik yang handal berfungsi sebagai pusat informasi tentang (prediksi) khasiat tanaman obat.
Selain itu, kode web-server akan disediakan dalam bentuk open-source library dengan dokumentasi yang rapi dan lengkap sehingga dapat dimanfaatkan oleh peneliti.
Hal ini akan mendukung kesinambungan pengembangan dan perawatan sistem.
Kode tersebut meliputi fungsi-fungsi perhitungan kernel gabungan, proses prediksi, serta penjelajahan (crawling) database terkait.
\end{enumerate}
