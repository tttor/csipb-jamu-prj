%
% Template Laporan Skripsi/Thesis
%
% @author  Andreas Febrian, Lia Sadita
% @version 1.03
%
% Dokumen ini dibuat berdasarkan standar IEEE dalam membuat class untuk
% LaTeX dan konfigurasi LaTeX yang digunakan Fahrurrozi Rahman ketika
% membuat laporan skripsi. Konfigurasi yang lama telah disesuaikan dengan
% aturan penulisan thesis yang dikeluarkan UI pada tahun 2008.
%

%
% Tipe dokumen adalah report dengan satu kolom.
%
\documentclass[12pt, a4paper, onecolumn, oneside, final]{report}

% Load konfigurasi LaTeX untuk tipe laporan thesis
\usepackage{uithesis}
\usepackage{natbib}
\usepackage{import}

\usepackage{subcaption}
\usepackage{rotating}
\usepackage{tikz}
\usepackage{amsmath}
\usepackage{amsfonts}
\usepackage{amssymb}
\usepackage[]{algorithm2e}
\usepackage{url}
\usepackage{nameref}
\usepackage{cleveref}
% \usepackage{algorithm}
\usepackage{framed}
\usepackage{listings}
\usepackage{rotating}
\usepackage{tikz}
\usepackage{pdflscape}

\usepackage{graphicx} % for pdf, bitmapped graphics files
\graphicspath{{./pics/}}
\DeclareGraphicsExtensions{.pdf,.jpeg,.png,.jpg}
% \usepackage{subfigure}

% Load konfigurasi khusus untuk laporan yang sedang dibuat
%-----------------------------------------------------------------------------%
% Informasi Mengenai Dokumen
%-----------------------------------------------------------------------------%
% 
% Judul laporan. 
\var{\judul}{Pemilihan Segmen dari Berbagai Segmentasi untuk Segmentasi Semantik Berdasarkan \textit{Conditional Random Field} melalui \textit{Latent Dirichlet Allocation} dan Algoritme Genetika}% 

% Tulis kembali judul laporan, kali ini akan diubah menjadi huruf kapital
\Var{\Judul}{Pemilihan Segmen dari Berbagai Segmentasi untuk Segmentasi Semantik Berdasarkan \textit{Conditional Random Field} melalui \textit{Latent Dirichlet Allocation} dan Algoritme Genetika}%
 
% Tulis kembali judul laporan namun dengan bahasa Ingris
\var{\judulInggris}{Selecting Segments from Multiple Segmentatation for Conditional Random Field based Semantic Segmentation via Latent Dirichlet Allocation and Genetic Algorithm}

% 
% Tipe laporan, dapat berisi Skripsi, Tugas Akhir, Thesis, atau Disertasi
\var{\type}{Skripsi}
% 
% Tulis kembali tipe laporan, kali ini akan diubah menjadi huruf kapital
\Var{\Type}{Skripsi}
% 
% Tulis nama penulis 
\var{\penulis}{Jogie Chandra}
% 
% Tulis kembali nama penulis, kali ini akan diubah menjadi huruf kapital
\Var{\Penulis}{Jogie Chandra}
% 
% Tulis NPM penulis
\var{\npm}{1106053640}
% 
% Tuliskan Fakultas dimana penulis berada
\Var{\Fakultas}{Ilmu Komputer}
\var{\fakultas}{Ilmu Komputer}
% 
% Tuliskan Program Studi yang diambil penulis
\Var{\Program}{Ilmu Komputer}
\var{\program}{Ilmu Komputer}
\var{\programEng}{Computer Science}
% 
% Tuliskan tahun publikasi laporan
\Var{\bulanTahun}{Desember 2014}
% 
% Tuliskan gelar yang akan diperoleh dengan menyerahkan laporan ini
\var{\gelar}{Sarjana Ilmu Komputer}
% 
% Tuliskan tanggal pengesahan laporan, waktu dimana laporan diserahkan ke 
% penguji/sekretariat
\var{\tanggalPengesahan}{24 Desember 2014} 
% 
% Tuliskan tanggal keputusan sidang dikeluarkan dan penulis dinyatakan 
% lulus/tidak lulus
\var{\tanggalLulus}{8 Januari 2015}
% 
% Tuliskan pembimbing 
\var{\pembimbing}{Wisnu Jatmiko, S.T., M.Kom., Dr. Eng.}
%\var{\pembimbingdua}{Dia S.Kom, M.Kom}
% 
% Alias untuk memudahkan alur penulisan paa saat menulis laporan
\var{\saya}{Penulis}

%-----------------------------------------------------------------------------%
% Judul Setiap Bab
%-----------------------------------------------------------------------------%
% 
% Berikut ada judul-judul setiap bab. 
% Silahkan diubah sesuai dengan kebutuhan. 
% 
\Var{\kataPengantar}{Kata Pengantar}
\Var{\babSatu}{Pendahuluan}
\Var{\babDua}{Landasan Teori}
\Var{\babTiga}{Metode Usulan}
% \Var{\babEmpat}{Implementasi}
\Var{\babEmpat}{Hasil Eksperimen dan Analisis}
\Var{\babLima}{Kesimpulan}
% Daftar pemenggalan suku kata dan istilah dalam LaTeX
\include{hype.indonesia}
% Daftar istilah yang mungkin perlu ditandai
\input{istilah}

\usepackage[shortlabels]{enumitem}
\usepackage{lipsum}
\usepackage{titlesec}
\usepackage[us,12hr]{datetime}

\titleformat{\paragraph}
{\normalfont\normalsize\bfseries}{\theparagraph}{1em}{}
\titlespacing*{\paragraph}
{0pt}{3.25ex plus 1ex minus .2ex}{1.5ex plus .2ex}

%\usepackage[backend=bibtex]{biblatex}
%\addbibresource{bib.bib}

% Awal bagian penulisan laporan
\begin{document}
%
% Sampul Laporan
%
% Sampul Laporan

%
% @author  unknown
% @version 1.01
% @edit by Andreas Febrian
%

\begin{titlepage}
    \begin{center}
        \bo{
            PETUNJUK PENGGUNAAN\\
            IJAH WEBSERVER
        }
        \vspace*{2cm}

        
        \vspace*{3cm}

        % \begin{figure}
        %     \begin{center}
        %         \includegraphics[width=3.5cm]{pics/logo_ijah.png}
        %     \end{center}
        % \end{figure}    
        \vspace*{0cm}
        
        \bo{
            Oleh: \\
            Wisnu Kusuma, Vektor Dewanto, \emph{et al.}
        }
        \vspace*{1cm}

        \vspace*{5.0cm}

        % informasi mengenai fakultas dan program studi
        \bo{
            Tropical Biopharmaca Research Center \& Computer Science Department\\
            Bogor Agricultural University
        }
        \vspace*{1cm}

        \vspace*{1cm}
        \bo{
            \today~\currenttime
        }
    \end{center}
\end{titlepage}


%
% Gunakan penomeran romawi
\pagenumbering{roman}

%
% load halaman judul dalam
% \addChapter{HALAMAN JUDUL}
% \include{judul_dalam}

%
% setelah bagian ini, halaman dihitung sebagai halaman ke 2
\setcounter{page}{2}
%\setcounter{secnumdepth}{4}

% sementara ga usah
% load halaman pengesahan
% \addChapter{HALAMAN PENGESAHAN}
% \include{pengesahan}

% \singlespacing
% \addChapter{ABSTRAK}
% \include{abstrak}

% \addChapter{\kataPengantar}
% \include{pengantar}

%
%
% \addChapter{ABSTRACT}
% /home/tor/jamu/dropbox/wrt/tor/ijah-webserver-nar/src/own/abstract.tex

%
% Daftar isi, gambar, dan tabel
%
\tableofcontents
% \clearpage
% \listoffigures
% \clearpage
%\listoftables
%\clearpage
%\lstlistoflistings
%\clearpage

%
% Gunakan penomeran Arab (1, 2, 3, ...) setelah bagian ini.
%
\pagenumbering{arabic}

%
%
%
\onehalfspacing

%% This is an example first chapter.  You should put chapter/appendix that you
%% write into a separate file, and add a line \include{yourfilename} to
%% main.tex, where `yourfilename.tex' is the name of the chapter/appendix file.
%% You can process specific files by typing their names in at the 
%% \files=
%% prompt when you run the file main.tex through LaTeX.
\chapter{Pendahuluan}

\section{Apa Itu Ijah Webserver?}

	\subsection{Definisi}
	Ijah Webserver merupakan sebuah webserver yang menyajikan pencarian dan/atau prediksi terhadap hubungan antara tanaman dan khasiatnya pada suatu penyakit. Informasi yang disajikan berupa hubungan antara tanaman dengan senyawa (Plant-Compound), senyawa dengan bioprotein (Compound-Protein), dan bioprotein dengan penyakit (Protein-Disease). Selain pencarian dari \emph{database}, Ijah juga melakukan prediksi terhadap hubungan yang belum diketahui menggunakan beberapa jenis algoritme, dan memberikan hasil berupa tingkat kemungkinan kebenaran prediksi. Hasil keterhubungan ini (baik hasil pencarian maupun prediksi) disajikan dalam beberapa jenis keluaran, yang utama yaitu visualisasi dalam bentuk graf \emph{multi-partite}. Tujuan Ijah Webserver adalah menyederhanakan dan membantu proses penelitian obat herbal atau Jamu dengan memberikan informasi potensi tanaman.

	\subsection{Kasus-penggunaan \emph{(use-case)}}
	Ada beberapa kasus-penggunaan \emph{(use-case)} dalam pencarian saat menggunakan Ijah Webserver.
		\subsubsection{Use-case 1 (Both drug-and-target inputs)} \label{sssec:end to end}
		\emph{Use-case} ini melibatkan kedua sisi input yaitu \emph{Drug-side} dan \emph{Target-side}. pada \emph{use case} ini, input dari kedua sisi harus ada. Pencarian ini bersifat end-to-end, misalkan Plant to Disease, dimana kita memasukkan beberapa jenis tanaman dari sisi Drug-side dan beberapa jenis penyakit dari Target-side, lalu hasilnya akan menunjukkan senyawa apa dalam tanaman tersebut yang dapat menarget protein yang ada dalam penyakit tersebut, sehingga berkhasiat menyembuhkan. Variasi input pada \emph{use-case} ini meliputi \textbf{Plant-Protein}, \textbf{Plant-Disease}, \textbf{Compound-Protein}, dan \textbf{Compound-Disease}.
		\subsubsection{Use-case 2 (Drug-side inputs only)} \label{sssec:drug only}
		\emph{Use-case} ini melibatkan hanya satu sisi yaitu \emph{Drug-side} saja. pada \emph{use-case} ini, input dari salah satu \emph{drug-side} (Plant atau Compound) harus ada. Sifat pencarian ini, misal kita memasukkan tanaman saja (Plant Search Only), maka semua senyawa yang terkandung dalam tanaman itu akan muncul, lalu protein yang terkait dengan senyawa tersebut, dan penyakit yang terkait dengan protein tersebut. Sederhananya semua yang terkait dengan tanaman yang diinputkan. Variasi input pada \emph{use-case} ini meliputi \textbf{Plant Search Only} dan \textbf{Compound Search Only} yang pada dasarnya sama dengan Plant Search.
		\subsubsection{Use-case 3 (Target-side inputs only)} \label{sssec:target only}
		\emph{Use case} ini melibatkan hanya satu sisi yaitu \emph{Target-side} saja. pada \emph{use case} ini, input dari salah satu \emph{Target-side} (Protein atau Disease) harus ada. Sifat pencarian ini mirip dengan Drug inputs only, namun bedanya dari sisi Target. Misal pada pencarian dari penyakit (Disease Search Only) akan muncul semua yang berkaitan dengan penyakit itu, termasuk tanaman yang berhubungan. Variasi input pada \emph{use-case} ini meliputi \textbf{Protein Search Only} dan \textbf{Disease Search Only}.


	\subsection{Menu-menu} %brief description about the menus
	Pada Ijah Webserver sendiri terdapat beberapa menu yang bisa dipilih:
	\begin{itemize}
	\item Home
	\item Manual
	\item Downloads
	\item Help/FAQ
	\item IjahV1
	\item Disclaimer
	\item ContactUs
	\item AboutUs
	\end{itemize}
	Penjelasan tentang menu-menu ini ada pada bagian \textbf{\nameref{chap:Menu}}


\section{Tampilan Awal \emph{(Home Page)}}
	 
	\subsection{Overview}
		\subsubsection{URL} \label{ssec:URL ijah}
		Ijah Webserver dapat diakses pada alamat URL \href{http://ijah.apps.cs.ipb.ac.id}{\textbf{http://ijah.apps.cs.ipb.ac.id}}

		\subsubsection{Home Page} %view home + penunjuk bagian
		
		\begin{figure}[!h]
		\centering
		\includegraphics[scale=0.3]{ijah_home_overview.png}
		\caption{Halaman utama Ijah Webserver:
		(1) \hyperref[ssec:URL ijah]{URL Ijah};
		(2) \hyperref[sssec:drug input]{Kotak Masukan \emph{Drug-side}};
		(3) \hyperref[sssec:target input]{Kotak Masukan \emph{Target-side}};
		(4) \hyperref[sssec:search]{Tombol Eksekusi (Search/Search and Predict)};
		(5) \hyperref[sssec:reset]{Tombol Reset};
		(6) \hyperref[ssec:example button]{Deretan Tombol Contoh};
		(7) \hyperref[ssec:menu list]{Deretan Menu};
		(8) \hyperref[ssec:judul]{Judul};
		(9) \hyperref[ssec:judul]{Logo Ijah Webserver};
		(10) \hyperref[ssec:judul]{\emph{Copyright} dan versi Ijah};
		(11) \hyperref[ssec:visitor log]{\emph{Visitor Log}}}
		\label{fig:ijah_home_overview}
		\end{figure}

	\subsection{Kotak Masukan \emph{(Input Fields)}}
		\subsubsection{Drug-side} \label{sssec:drug input}
		Pada Drug-side, input terdiri dari tanaman (Plant) dan senyawa (Compound). Untuk menjaga konsistensi, ketika salah satu dipilih, maka pilihan lain akan menghilang, misal kita telah memilih untuk menginputkan Plant, maka Compound tidak bisa dipilih, begitu pula sebaliknya. Jika ingin mengembalikan/mengosongkan kondisi input kembali, gunakan tombol \hyperref[reset]{\textbf{Reset}}.
			\paragraph{Plants}
			Input Plant terdiri dari nama Latin dan nama Indonesia dari tanaman, yang dipisahkan oleh suatu garis vertikal. Hal ini memudahkan jika kita hanya mengetahui nama latin suatu tanaman dan tidak mengetahui nama lokalnya, begitu pula sebaliknya jika hanya mengetahui nama lokalnya. Selain membantu pencarian, pada kasus salah satu nama tidak diketahui maka akan didapat pengetahuan mengenai nama tanaman tersebut.
			\paragraph{Compounds}
			Input Compound terdiri dari ID CAS, \hyperref[knapsack]{ID KNApSAcK}, atau \hyperref[drugbank]{ID DrugBank} dari senyawa tersebut, setiap ID dipisahkan oleh garis vertikal. 
		\subsubsection{Target-side} \label{sssec:target input}
		Pada Target-side, input terdiri dari Protein dan penyakit (Disease). Untuk menjaga konsistensi, ketika salah satu dipilih, maka pilihan lain akan menghilang, misal kita telah memilih untuk menginputkan Protein, maka Disease tidak bisa dipilih, begitu pula sebaliknya. Jika ingin mengembalikan/mengosongkan kondisi input kembali, gunakan tombol \hyperref[reset]{\textbf{Reset}}.
			\paragraph{Protein}
			Input Protein terdiri dari \hyperref[uniprot]{nama Uniprot atau ID Uniprot}. Nama dan ID dipisahkan oleh garis vertikal.
			\paragraph{Disease}
			Input Disease terdiri dari \hyperref[omim]{nama OMIM atau ID OMIM}. Nama dan ID dipisahkan oleh garis vertikal.

	\subsection{Tombol Eksekusi}
	Setelah memberikan input, ada beberapa aksi yang dapat dilakukan, yaitu meneruskan pencarian dengan tombol Search atau mengosongkan semua input dengan tombol Reset.
		\subsubsection{Search (Search and Predict)} \label{sssec:search}
		Bergantung pada jenis input yang diberikan, tombol ini dapat menjalankan fungsi Search saja atau Search and Predict. Lengkapnya lihat pada bagian \hyperref[process]{\textbf{Proses}}.
		\subsubsection{Reset} \label{sssec:reset}
		Tombol Reset ini \textbf{memiliki fungsi yang cukup penting}. Saat ini terdapat kekurangan dimana saat menghapus input, input tidak benar-benar terhapus namun telah tersimpan untuk diproses. Sementara kekurangan ini masih diusahakan untuk diatasi, \textbf{\emph{sangat dianjurkan}} menggunakan tombol Reset ini untuk menghapus input ke kondisi kosong semula.

	\subsection{Deretan Tombol Contoh \emph{(Example)}} \label{ssec:example button}
	Deretan tombol contoh ini merupakan kumpulan contoh \emph{use-case} yang ada pada Ijah Webserver. Jadi selain menggunakan input manual, tombol-tombol contoh ini dapat memudahkan dalam menuju \emph{use-case} tertentu. Kumpulan contoh ini terbagi menjadi tiga kelompok, dimana Example \#1a - \#1d merupakan \emph{use-case} \hyperref[end to end]{\emph{Both Drug and Target inputs}}, Example \#2a - \#2b merupakan \emph{use-case} \hyperref[drug only]{\emph{Drug inputs only}}, dan Example \#3a - \#3b merupakan \emph{use-case} \hyperref[target only]{\emph{Target inputs only}}. Lebih lengkapnya:

	\begin{itemize}
	\item \textbf{Example \#1a} - Plant-Disease
	\item \textbf{Example \#1b} - Plant-Protein
	\item \textbf{Example \#1c} - Compound-Protein
	\item \textbf{Example \#1d} - Compound-Disease
	\item \textbf{Example \#2a} - Plant search only
	\item \textbf{Example \#2b} - Compound search only
	\item \textbf{Example \#3a} - Protein search only
	\item \textbf{Example \#3b} - Disease search only
	\end{itemize}

	\subsection{Deretan Menu} \label{ssec:menu list}
	Pada bagian ini ada sederetan menu yang dapat di-klik. Penjelasan lebih lanjut tentang menu ini ada pada bagian \textbf{\nameref{Menu}}.

	\subsection{Judul, Versi, Logo, dan \emph{Copyright} Ijah} \label{ssec:judul}
	Pada bagian tengah halaman terdapat judul webserver ini, yaitu \textbf{``Ijah Webserver -- Search for (Plant-Compound) -- (Protein-Disease) Connectivity''}. Logo Ijah terdapat pada bagian kiri atas. Pada bagian bawah kiri halaman terdapat bagian yang berisi \emph{copyright} dan versi Ijah Webserver yang berjalan. Di bagian ini juga ada saran penggunaan untuk pengalaman terbaik saat menggunakan Ijah Webserver. Pada nomor versi Ijah Webserver, 8 digit pertama merupakan format tahun-bulan-tanggal dan 4 digit terakhir merupakan jam dan menit. Sebagai contoh di Ijah Webserver v201702231639 berarti versi ini di-\emph{deploy} pada tanggal 23 Februari 2017 pukul 16:39. Pada \emph{copyright} dijelaskan bahwa hak cipta Ijah Webserver ada pada Departemen Ilmu Komputer IPB dan Pusat Riset Biofarmaka IPB.


	\subsection{Visitor Log} \label{ssec:visitor log}
	Visitor Log terdapat di sebelah kanan bawah, setelah \emph{copyright} dan versi. Visitor Log adalah fitur yang menghitung jumlah kunjungan Ijah Webserver beserta negara asal kunjungan. Area Visitor Log ini dapat diklik untuk melihat statistik kunjungan lebih lengkap, seperti pada gambar \ref{fig:ijah_flagcount}

	\begin{figure}[H]
	\centering
	\includegraphics[scale=0.3]{ijah_flagcount.png}
	\caption{Statistik lebih detail yang akan muncul setelah mengklik area Visitor Log}
	\label{fig:ijah_flagcount}
	\end{figure}

\section{Repositori \emph{Source Code}} \label{sec:ws_source}
\emph{Source Code} untuk Ijah Webserver dapat diakses di repositori GitHub di alamat URL \href{https://github.com/tttor/csipb-jamu-prj}{\textbf{https://github.com/tttor/csipb-jamu-prj}}.

Pada repositori ini terdapat beberapa folder untuk setiap \emph{source code}

\begin{itemize}
\item \textbf{Webserver}\\
\href{https://github.com/tttor/csipb-jamu-prj/tree/master/webserver}{\textbf{https://github.com/tttor/csipb-jamu-prj/tree/master/webserver}}.
\item \textbf{Predictor}\\
\href{https://github.com/tttor/csipb-jamu-prj/tree/master/predictor}{\textbf{https://github.com/tttor/csipb-jamu-prj/tree/master/predictor}}.
\item \textbf{Database} (lihat bab \ref{db chapter} )\\
\href{https://github.com/tttor/csipb-jamu-prj/tree/master/database}{\textbf{https://github.com/tttor/csipb-jamu-prj/tree/master/database}}.
\end{itemize}

%% This is an example first chapter.  You should put chapter/appendix that you
%% write into a separate file, and add a line \include{yourfilename} to
%% main.tex, where `yourfilename.tex' is the name of the chapter/appendix file.
%% You can process specific files by typing their names in at the 
%% \files=
%% prompt when you run the file main.tex through LaTeX.
\chapter{Rumusan Masalah}
% Rumusan masalah mencakup :
%  Penjelasan mengenai masalah yang menjadi topik penelitian
%  Cakupan dan batasan penelitian
%  Esensi penelitian yang dilakukan

\section{Penjelasan masalah}
Masalah-masalah yang akan dipecahkan dalam penelitian ini meliputi:

\begin{enumerate} [topsep=0mm]
\itemsep0mm
\item bagaimana mengukur kesamaan dalam bentuk kernel dari dua data dalam ruang kimia, genomik atau farmakologi?
\item bagaimana menggabungkan informasi dari tiga ruang: kimia, genomik dan farmakologi?
\item bagaimana merumuskan metode prediksi berbasis kernel untuk estimasi interaksi senyawa-protein yang akurat, 
(dengan akurasi lebih besar dari 95\%)?
\item bagaimana memilah dan mengunduh secara otomatis informasi dari banyak database yang relevan?
\item bagaimana desain antar-muka yang user-friendly untuk aplikasi-web yang melayani estimasi interaksi senyawa-protein?
\item bagaimana desain web-server sehingga memiliki tingkat utility, maintainability dan scalability yang tinggi?
\end{enumerate}

\section{Cakupan dan batasan penelitian}
Cakupan dan batasan dari penelitian ini adalah sebagai berikut:

\begin{enumerate} [topsep=0mm]
\itemsep0mm
\item metode prediksi berbasis kernel
\item menggunakan kernel dari tiga ruang: kimia, genomik dan farmakologi
\item web-server menerima input berikut: 
	\begin{itemize} [topsep=0mm]
	\itemsep0mm
	\item dari aspek obat: nama tanaman atau nama senyawa
	\item dari aspek penyakit: nama penyakit atau nama protein
	\end{itemize}
\item informasi untuk prediksi berasal dari database publik yang tersedia
\item prediksi mencakup senyawa organik dari tanaman herbal dan senyawa sintetik dari obat sintetik
\end{enumerate}

\section{Esensi penelitian}
Esensi dari penelitian ini meliputi:

\begin{enumerate} [topsep=0mm]
\itemsep0mm
\item
Prediksi yang akurat terhadap interaksi senyawa bioaktif dengan protein merupakan informasi yang krusial untuk pengembangan obat dan suplemen makanan, baik herbal maupun sintetis.
Dengan informasi ini sebagai penapis in-silico, uji klinis yang mahal dan lama hanya dilakukan pada kandidat senyawa yang berpotensi tinggi.

\item
Web-server publik yang handal berfungsi sebagai pusat informasi tentang (prediksi) khasiat tanaman obat.
Selain itu, kode web-server akan disediakan dalam bentuk open-source library dengan dokumentasi yang rapi dan lengkap sehingga dapat dimanfaatkan oleh peneliti.
Hal ini akan mendukung kesinambungan pengembangan dan perawatan sistem.
Kode tersebut meliputi fungsi-fungsi perhitungan kernel gabungan, proses prediksi, serta penjelajahan (crawling) database terkait.
\end{enumerate}

%% This is an example first chapter.  You should put chapter/appendix that you
%% write into a separate file, and add a line \include{yourfilename} to
%% main.tex, where `yourfilename.tex' is the name of the chapter/appendix file.
%% You can process specific files by typing their names in at the 
%% \files=
%% prompt when you run the file main.tex through LaTeX.
\chapter{Menu-menu}  \label{Menu}

\section{Manual}
Menu \emph{Manual} berisi \emph{link} untuk mendownload file Manual penggunaan Ijah Webserver. Kedepannya, isi file Manual juga akan ditampilkan di halaman ini.

\begin{figure}[H]
	\centering
	\includegraphics[scale=0.3]{ijah_manual_page.png}
	\caption{Halaman Manual}
	\label{fig:ijah_manual_page}
\end{figure}

\section{Downloads} \label{Downloads}
Menu \emph{Downloads} menyediakan link untuk mengunduh Metadata seluruh item (tanaman, senyawa, protein, dan penyakit) yang ada dalam database Ijah Webserver. Juga disediakan data seluruh konektivitas (keterhubungan) antar item dan data lainnya seperti similarity data, data sekuens protein, dan lain lain.

Untuk mengunduh file pada halaman ini cukup klik nama file yang ingin diunduh

\begin{figure}[H]
	\centering
	\includegraphics[scale=0.3]{ijah_downloadpage.png}
	\caption{Daftar file yang dapat di\-download}
	\label{fig:ijah_downloadpage}
\end{figure}

	\subsection{Metadata}
	File yang dapat didownload pada kategori Metadata yaitu:

	\begin{itemize}
	\item \textbf{plant\_metadata.txt} -- Berisi metadata seluruh tanaman pada \emph{database} Ijah Webserver. 
	\item \textbf{compound\_metadata.txt} -- Berisi metadata seluruh senyawa (\emph{compound}) pada \emph{database} Ijah Webserver.
	\item \textbf{protein\_metadata.txt} -- Berisi metadata seluruh bioprotein pada \emph{database} Ijah Webserver.
	\item \textbf{disease\_metadata.txt} -- Berisi metadata seluruh penyakit pada \emph{database} Ijah Webserver.
	\end{itemize}

	\subsection{Connectivity Data}
	File yang dapat didownload pada kategori Connectivity Data yaitu:

	\begin{itemize}
	\item \textbf{plant\_vs\_compound\_connectivity.txt} -- Berisi data konektivitas tanaman dengan senyawa beserta skor konektivitasnya pada \emph{database} Ijah Webserver.
	\item \textbf{compound\_vs\_protein\_connectivity.txt} -- Berisi data konektivitas senyawa dengan protein beserta skor konektivitasnya pada \emph{database} Ijah Webserver.
	\item \textbf{protein\_vs\_disease\_connectivity.txt} -- Berisi data konektivitas protein dengan penyakit beserta skor konektivitasnya pada \emph{database} Ijah Webserver.
	\end{itemize}

	\subsection{Similarity (kernels) data}
	\begin{itemize}
	\item compound\_similarity\_simcomp.txt  
	\item protein\_similarity\_smithwatermann.txt  
	\end{itemize}

	\subsection{Feature data}
	\begin{itemize}
	\item compound\_feature\_smiles.txt  
	\item protein\_feature\_sequence.txt  
	\end{itemize}

	\subsection{Documents}
	Dokumentasi Ijah Webserver
	\begin{itemize}
	\item \textbf{ijah\_webserver\_manual.pdf} -- Manual penggunaan Ijah Webserver
	\item \textbf{ijah\_webserver\_paper.pdf} -- Paper penelitian Ijah Webserver
	\end{itemize}

\section{Help/FAQ}
Menu \emph{Help/FAQ} berisikan beberapa pertanyaan umum \emph{(Frequently Asked Questions)} beserta jawabannya. Secara \emph{default} bahasa yang tersedia adalah bahasa Inggris namun dapat diubah ke bahasa Indonesia dengan mengklik tombol <todo>

\begin{figure}[H]
	\centering
	\includegraphics[scale=0.3]{ijah_faq.png}
	\caption{Halaman Help/FAQ}
	\label{fig:ijah_faq}
\end{figure}

Anda dapat menambahkan pertanyaan melalui menu \textbf{\nameref{Contact Us}}.

\section{IjahV1}
Menu \emph{IjahV1} berisi link menuju versi awal Ijah (Ijah versi 1, atau IjahV1)

\begin{figure}[H]
	\centering
	\includegraphics[scale=0.3]{ijah_v1.png}
	\caption{Isi halaman menu IjahV1}
	\label{fig:ijah_v1}
\end{figure}

Jika anda ingin mengakses IjahV1 silakan mengunjungi \href{http://ijah.apps.cs.ipb.ac.id/ijahv1/}{\textbf{http://ijah.apps.cs.ipb.ac.id/ijahv1/}}

\section{Disclaimer}

Menu \emph{Disclaimer} berisikan pernyataan batasan \emph{responsibility} pihak Ijah Webserver atas penggunaan hasil \emph{output} dari Ijah Webserver.

\begin{figure}[H]
	\centering
	\includegraphics[scale=0.3]{ijah_disclaimer.png}
	\caption{Isi halaman Disclaimer}
	\label{fig:ijah_disclaimer}
\end{figure}

Terjemahan dari isi halaman Disclaimer yaitu:

\textit{``Ijah Webserver ini gratis, dengan harapan dapat membantu banyak pihak. Namun kami tidak menjamin akibat dari penggunaan informasi dari Webserver ini. Dan kami tidak bertanggungjawab atas insiden atau kerusakan baik langsung maupun tidak langsung yang diakibatkan oleh penggunaan data atau informasi yang kami sediakan di Webserver ini.''}


\section{ContactUs} \label{Contact Us}

Menu \emph{ContactUs} merupakan sarana komunikasi antara pengguna dengan tim pengembang Ijah Webserver. Dengan mengisikan tanggapan/keluhan/saran pada form Contact Us, tanggapan anda akan terkirim ke E-mail kami.

\begin{figure}[H]
	\centering
	\includegraphics[scale=0.3]{ijah_contact_page.png}
	\caption{Halaman Contact Us pada Ijah Webserver:
	(1) \hyperref[subject]{Pemilihan Subject};
	(2) \hyperref[message]{Message box};
	(3) \hyperref[personal data]{Name box};
	(4) \hyperref[personal data]{E-mail box};
	(5) \hyperref[personal data]{Affiliation box};
	(6) \hyperref[submit]{Tombol Submit};
	}
	\label{fig:ijah_contact_page}
\end{figure}

	\subsection{Subject} \label{subject}
	Pada bagian \emph{Subject} anda akan memilih jenis tanggapan anda 
	\begin{itemize}
	\item \textbf{Question:} mengajukan pertanyaan.
	\item \textbf{Error/Bugs:} memberitahukan adanya kesalahan.
	\item \textbf{Wishlist:} menyampaikan saran fitur apa yang diinginkan pada versi selanjutnya.
	\item \textbf{Complaint:} menyampaikan keluhan dalam penggunaan.
	\item \textbf{Suggestion:} memberikan saran.
	\item \textbf{Testimony:} memberikan testimoni tentang Ijah Webserver.
	\item \textbf{Other:} untuk menyampaikan hal lain yang tidak termasuk dalam kategori diatas.
	\end{itemize}

	\subsection{Message} \label{message}
	Isikan pesan anda pada bagian \emph{Message}. Jumlah karakter tidak dibatasi, jadi mohon untuk tidak menggunakan singkatan jika tidak diperlukan demi memudahkan tim kami dalam membaca tanggapan anda. 


	\subsection{Memasukkan Data Diri Anda} \label{personal data}
	Setelah menuliskan tanggapan, silakan isi data diri anda, nama pada bagian \emph{Name}, alamat E\-mail anda pada bagian \emph{E\-mail}, dan afiliasi anda (organisasi, universitas, atau perusahaan) pada bagian \emph{Affiliation}.

	\textbf{Penting:} Nama dan alamat E-mail \emph{wajib} diisi. 

	\subsection{Mengirim Tanggapan} \label{submit}
	Setelah semua data terisi lengkap, tekan \textbf{Submit} dan tanggapan anda akan terkirim.

\section{AboutUs}

Menu \emph{AboutUs} berisi info tentang tim pengembang Ijah Webserver

\begin{figure}[H]
	\centering
	\includegraphics[scale=0.3]{ijah_about.png}
	\caption{Isi halaman About Us}
	\label{fig:ijah_about}
\end{figure}

Tim pengembang Ijah Webserver yaitu:

\begin{itemize}
\item Principal Investigator:\\ 
Dr. Wisnu Ananta Kusuma

\item System-Integration and Lead Programmer:\\
Vektor Dewanto

\item Programmers:\\
\# Webserver:
\begin{itemize}
\item Haekal Zidni Barkan
\item Ivan Maulana
\item Royan Hudayana
\item Isnan Mulia
\item Yudha Kristanto (Ijahv1)
\item Aries Fitriawan (Ijahv1)
\end{itemize}
\# Database:
\begin{itemize}
\item Fahmi Amir
\item Yogi Sumantri
\item Hartanto Tantriawan
\end{itemize}
\# Predictor:
\begin{itemize}
\item Ajmal Kurnia
\end{itemize}
\end{itemize}

% \emph{Use case} ini melibatkan hanya satu sisi yaitu \emph{Drug-side} saja. pada \emph{use case} ini, input dari salah satu \emph{drug-side} (Plant atau Compound) harus ada.

% \section{Plant Search Only}

% Pada contoh ini, input dari \emph{drug-side} berupa tanaman (Plant). Contoh ini mencari dari tanaman yang diinputkan, apa saja senyawa yang terkandung dalam tanaman itu, dan senyawa tersebut dapat berkhasiat untuk penyakit apa melalui protein apa.

% \subsection{Input}
% \begin{figure}[H]
% 	\centering
% 	\includegraphics[scale=0.3]{example_2a.png}
% 	\caption{Contoh \emph{use case} input Plant Search Only}
% 	\label{fig:example_2a}
% \end{figure}

% Seperti yang telah dibahas pada bab sebelumnya, untuk jenis \emph{use case} ini tombol hanya bertuliskan \textbf{Search}, bukan Search and Predict seperti \emph{use case} sebelumnya.

% \subsection{Output}
% Output pada contoh \emph{search} dari 3 jenis tanaman ini adalah sebagai berikut, dimulai dari \emph{Connectivity Graph Output}

% \begin{figure}[H]
% 	\centering
% 	\includegraphics[scale=0.3]{ijah_example_2a_graph.png}
% 	\caption{Connectivity Graph Output pada use case Plant Search Only}
% 	\label{fig:ijah_example_2a_output}
% \end{figure}

% Graf yang dihasilkan kali ini cukup besar karena menampilkan semua penyakit yang terkait dengan seluruh senyawa yang dikandung tanaman-tanaman tersebut. Karena contoh ini menampilkan semua yang terkait dengan tanaman yang diinputkan.

% Hasil \emph{Connectivity Text Output} untuk contoh ini:

% \begin{figure}[H]
% 	\centering
% 	\includegraphics[scale=0.3]{ijah_example_2a_text.png}
% 	\caption{Connectivity Text Output pada use case Plant Search Only}
% 	\label{fig:ijah_example_2a_text}
% \end{figure}

% Hasil \emph{Metadata Text Output} untuk contoh ini:

% \begin{figure}[H]
% 	\centering
% 	\includegraphics[scale=0.3]{ijah_example_2a_meta.png}
% 	\caption{Metadata Text Output pada use case Plant Search Only}
% 	\label{fig:ijah_example_2a_meta}
% \end{figure}

% \section{Compound Search Only}
% Contoh lain dari \emph{use case} Drug-Side Only yaitu Compound Search Only dimana perbedaannya hanya pada input, yaitu menginputkan senyawa (Compound) untuk mencari semua (Plant, Protein, Disease) yang terkait dengan senyawa yang diinputkan.


%% This is an example first chapter.  You should put chapter/appendix that you
%% write into a separate file, and add a line \include{yourfilename} to
%% main.tex, where `yourfilename.tex' is the name of the chapter/appendix file.
%% You can process specific files by typing their names in at the 
%% \files=
%% prompt when you run the file main.tex through LaTeX.
\chapter{Use Case 3 - Target-Side Only Input}

\emph{Use case} ini melibatkan hanya satu sisi yaitu \emph{Target-side} saja. pada \emph{use case} ini, input dari salah satu \emph{Target-side} (Protein atau Disease) harus ada.

\section{Protein Search Only}

Pada contoh ini, input dari \emph{Target-side} berupa protein. Contoh ini mencari dari protein yang diinputkan, apa saja senyawa dan penyakit yang terkait dengan protein itu, dan tanaman apa yang dapat menarget protein itu dengan senyawa terkait yang dikandungnya.

\subsection{Input}
\begin{figure}[H]
	\centering
	\includegraphics[scale=0.3]{example_3a.png}
	\caption{Contoh \emph{use case} input Protein Search Only}
	\label{fig:example_3a}
\end{figure}

Seperti yang telah dibahas pada bab Use Case 1, untuk jenis \emph{use case} ini tombol hanya bertuliskan \textbf{Search}, bukan Search and Predict seperti \emph{use case} pertama.

\subsection{Output}
Output pada contoh \emph{search} dari 2 jenis protein ini adalah sebagai berikut, dimulai dari \emph{Connectivity Graph Output}

\begin{figure}[H]
	\centering
	\includegraphics[scale=0.3]{ijah_example_3a_graph.png}
	\caption{Connectivity Graph Output pada use case Protein Search Only}
	\label{fig:ijah_example_3a_output}
\end{figure}

Graf yang dihasilkan kali ini lebih besar lagi karena menampilkan semua senyawa yang terkait dengan protein yang diinputkan, dan dalam contoh ini sangat banyak senyawa yang dapat menarget protein ini. Karena contoh ini menampilkan semua yang terkait dengan protein yang diinputkan.

Hasil \emph{Connectivity Text Output} untuk contoh ini:

\begin{figure}[H]
	\centering
	\includegraphics[scale=0.3]{ijah_example_3a_text.png}
	\caption{Connectivity Text Output pada use case Protein Search Only}
	\label{fig:ijah_example_3a_text}
\end{figure}

Hasil \emph{Metadata Text Output} untuk contoh ini:

\begin{figure}[H]
	\centering
	\includegraphics[scale=0.3]{ijah_example_3a_meta.png}
	\caption{Metadata Text Output pada use case Protein Search Only}
	\label{fig:ijah_example_3a_meta}
\end{figure}

\section{Disease Search Only}
Contoh lain dari \emph{use case} Target-Side Only yaitu Disease Search Only dimana perbedaannya hanya pada input, yaitu menginputkan penyakit (Disease) untuk mencari semua (Plant, Compound, Protein) yang terkait dengan penyakit yang diinputkan.




%% This is an example first chapter.  You should put chapter/appendix that you
%% write into a separate file, and add a line \include{yourfilename} to
%% main.tex, where `yourfilename.tex' is the name of the chapter/appendix file.
%% You can process specific files by typing their names in at the 
%% \files=
%% prompt when you run the file main.tex through LaTeX.
\chapter{Metode Prediksi} \label{chap:prediksi}

% Bab ini akan membahas deretan menu yang ada di bagian atas halaman Ijah Webserver.

% \begin{figure}[H]
% 	\centering
% 	\includegraphics[scale=0.3]{ijah_menu_top.png}
% 	\caption{Deretan Menu pada Ijah Webserver}
% 	\label{fig:ijah_menu_top}
% \end{figure}

% \section{Manual}

% Menu \emph{Manual} berisi \emph{link} untuk mendownload file Manual penggunaan Ijah Webserver. Kedepannya, isi file Manual juga akan ditampilkan di halaman ini.

% \begin{figure}[H]
% 	\centering
% 	\includegraphics[scale=0.3]{ijah_manual_page.png}
% 	\caption{Halaman Manual}
% 	\label{fig:ijah_manual_page}
% \end{figure}

% \section{Downloads} \label{Downloads}

% Menu \emph{Downloads} menyediakan link untuk mengunduh Metadata seluruh item (tanaman, senyawa, protein, dan penyakit) yang ada dalam database Ijah Webserver. Juga disediakan data seluruh konektivitas (keterhubungan) antar item dan data lainnya seperti similarity data, data sekuens protein, dan lain lain.

% Untuk mengunduh file pada halaman ini cukup klik nama file yang ingin diunduh

% \begin{figure}[H]
% 	\centering
% 	\includegraphics[scale=0.3]{ijah_downloadpage.png}
% 	\caption{Daftar file yang dapat di\-download}
% 	\label{fig:ijah_downloadpage}
% \end{figure}

% \subsection{Metadata}
% File yang dapat didownload pada kategori Metadata yaitu:

% \begin{itemize}
% \item \textbf{plant\_metadata.txt} -- Berisi metadata seluruh tanaman pada \emph{database} Ijah Webserver. 
% \item \textbf{compound\_metadata.txt} -- Berisi metadata seluruh senyawa (\emph{compound}) pada \emph{database} Ijah Webserver.
% \item \textbf{protein\_metadata.txt} -- Berisi metadata seluruh bioprotein pada \emph{database} Ijah Webserver.
% \item \textbf{disease\_metadata.txt} -- Berisi metadata seluruh penyakit pada \emph{database} Ijah Webserver.
% \end{itemize}

% \subsection{Connectivity Data}
% File yang dapat didownload pada kategori Connectivity Data yaitu:

% \begin{itemize}
% \item \textbf{plant\_vs\_compound\_connectivity.txt} -- Berisi data konektivitas tanaman dengan senyawa beserta skor konektivitasnya pada \emph{database} Ijah Webserver.
% \item \textbf{compound\_vs\_protein\_connectivity.txt} -- Berisi data konektivitas senyawa dengan protein beserta skor konektivitasnya pada \emph{database} Ijah Webserver.
% \item \textbf{protein\_vs\_disease\_connectivity.txt} -- Berisi data konektivitas protein dengan penyakit beserta skor konektivitasnya pada \emph{database} Ijah Webserver.
% \end{itemize}

% \subsection{Similarity (kernels) data}
% \begin{itemize}
% \item compound\_similarity\_simcomp.txt  
% \item protein\_similarity\_smithwatermann.txt  
% \end{itemize}

% \subsection{Feature data}
% \begin{itemize}
% \item compound\_feature\_smiles.txt  
% \item protein\_feature\_sequence.txt  
% \end{itemize}

% \subsection{Documents}
% Dokumentasi Ijah Webserver
% \begin{itemize}
% \item \textbf{ijah\_webserver\_manual.pdf} -- Manual penggunaan Ijah Webserver
% \item \textbf{ijah\_webserver\_paper.pdf} -- Paper penelitian Ijah Webserver
% \end{itemize}

% \section{Help/FAQ}
% Menu \emph{Help/FAQ} berisikan beberapa pertanyaan umum \emph{(Frequently Asked Questions)} beserta jawabannya. 

% \begin{figure}[H]
% 	\centering
% 	\includegraphics[scale=0.3]{ijah_faq.png}
% 	\caption{Halaman Help/FAQ}
% 	\label{fig:ijah_faq}
% \end{figure}

% Pertanyaan pertanyaan tersebut yaitu:

% \textbf{Q:} Bagaimana cara mengunduh semua data yang tersedia pada Ijah?

% A: Silakan lihat pada menu \nameref{Downloads}.

% \textbf{Q:} Bagaimana cara melaporkan isu atau galat?

% A: Silakan laporkan pada menu \nameref{Contact Us} atau pada \href{https://github.com/tttor/csipb-jamu-prj}{\emph{repository} publik kami}.

% \textbf{Q:} Bagaimana cara mendapatkan SQL dump dari database Ijah?

% A: Silakan kontak kami melalui menu \nameref{Contact Us}.

% \textbf{Q:} Apakah \emph{source code} Ijah tersedia secara publik?

% A: Ya, ada pada \href{https://github.com/tttor/csipb-jamu-prj}{\emph{repository} publik kami}.

% \textbf{Q:} Bagaimana cara mengutip dari paper Ijah Webserver?

% A: Silakan merujuk ke \href{http://ijah.apps.cs.ipb.ac.id/api/ijah_webserver_online.bib}{ijah\_webserver\_online.bib}.

% \textbf{Q:} Berapa banyak kunjungan ke Ijah Webserver?

% A: Silakan lihat \emph{visitor log} pada Footer kanan bawah, penghitungan log dimulai sejak 13 Januari 2017 pukul 18:00. Untuk lebih detail anda bisa mengklik area \emph{visitor log} tersebut.

% \textbf{Q:} Jika pertanyaan saya tidak ada pada daftar ini, dimana kami harus bertanya?

% A: Silakan kontak kami melalui menu \nameref{Contact Us}.

% \section{Ijah v1}

% Menu \emph{Ijah v1} berisi link menuju versi awal Ijah (Ijah versi 1, atau Ijah v1)

% \begin{figure}[H]
% 	\centering
% 	\includegraphics[scale=0.3]{ijah_v1.png}
% 	\caption{Isi halaman menu Ijah v1}
% 	\label{fig:ijah_v1}
% \end{figure}

% Jika anda ingin mengakses Ijah v1 silakan mengunjungi \url{http://ijah.apps.cs.ipb.ac.id/ijahv1/}

% \section{Disclaimer}

% Menu \emph{Disclaimer} berisikan pernyataan batasan responsibility pihak Ijah Webserver atas penggunaan hasil \emph{output} dari Ijah Webserver.

% \begin{figure}[H]
% 	\centering
% 	\includegraphics[scale=0.3]{ijah_disclaimer.png}
% 	\caption{Isi halaman Disclaimer}
% 	\label{fig:ijah_disclaimer}
% \end{figure}

% Di halaman ini kami menyatakan bahwa penggunaan Ijah Webserver ini gratis, dengan harapan dapat membantu banyak pihak. Namun kami tidak menjamin akibat dari penggunaan informasi dari Webserver ini. Dan kami tidak bertanggungjawab atas insiden atau kerusakan baik langsung maupun tidak langsung yang diakibatkan oleh penggunaan data atau informasi yang kami sediakan di Webserver ini.

% \section{Contact Us} \label{Contact Us}

% Menu \emph{Contact Us} merupakan sarana komunikasi antara pengguna dengan tim pengembang Ijah Webserver. Dengan mengisikan tanggapan/keluhan/saran pada form Contact Us, tanggapan anda akan terkirim ke E-mail kami.

% \begin{figure}[H]
% 	\centering
% 	\includegraphics[scale=0.3]{ijah_contact_page.png}
% 	\caption{Halaman Contact Us pada Ijah Webserver}
% 	\label{fig:ijah_contact_page}
% \end{figure}

% \subsection{Subject}
% Pada bagian \emph{Subject} anda akan memilih jenis tanggapan anda, yaitu pertanyaan (Question), memberitahukan adanya kesalahan (Error/Bugs), menyampaikan saran fitur apa yang diinginkan pada versi selanjutnya (Wishlist), menyampaikan keluhan (Complaint), saran (Suggestion), memberikan testimoni tentang Ijah Webserver (Testimony), atau menyampaikan hal lain yang tidak termasuk dalam kategori diatas, dapat dikategorikan pada Other.

% \begin{figure}[H]
% 	\centering
% 	\includegraphics[scale=0.3]{ijah_contact_subject.png}
% 	\caption{Bagian \emph{Subject} pada menu Contact Us}
% 	\label{fig:ijah_contact_subject}
% \end{figure}

% \subsection{Message}
% Isikan pesan anda pada bagian \emph{Message}. Jumlah karakter tidak dibatasi, jadi mohon untuk tidak menggunakan singkatan jika tidak diperlukan demi memudahkan tim kami dalam membaca tanggapan anda. 

% \begin{figure}[H]
% 	\centering
% 	\includegraphics[scale=0.3]{ijah_contact_message.png}
% 	\caption{Bagian \emph{Message} pada menu Contact Us}
% 	\label{fig:ijah_contact_message}
% \end{figure}

% \subsection{Memasukkan Data Diri Anda}
% Setelah menuliskan tanggapan, silakan isi data diri anda, nama pada bagian \emph{Name}, alamat E\-mail anda pada bagian \emph{E\-mail}, dan afiliasi anda (organisasi, universitas, atau perusahaan) pada bagian \emph{Affiliation}.

% \textbf{Penting:} Nama dan alamat E-mail \emph{wajib} diisi. 

% \begin{figure}[H]
% 	\centering
% 	\includegraphics[scale=0.3]{ijah_contact_nameinput.png}
% 	\caption{Form pengisian nama, E-mail, dan afiliasi}
% 	\label{fig:ijah_contact_nameinput}
% \end{figure}

% Setelah terisi, tekan \textbf{Submit} dan tanggapan anda akan terkirim.

% \subsection{About Us}

% Menu \emph{About Us} berisi info tentang tim pengembang Ijah Webserver

% \begin{figure}[H]
% 	\centering
% 	\includegraphics[scale=0.3]{ijah_about.png}
% 	\caption{Isi halaman About Us}
% 	\label{fig:ijah_about}
% \end{figure}
%\printbibliography
%
% Daftar Pustaka
%\include{pustaka}
%biblama (bukan biblatex)
% \bibliography{main, isnan}{}
%\bibliography{references}{}
%biblama (bukan biblatex)
% \bibliographystyle{apalikerd}
% \bibliographystyle{apa}
%\bibliographystyle{ieeetr}

\begin{singlespace}
	\bibliography{bibtex/drugtarget_pred,bibtex/database}
	% \bibliographystyle{apa}
	\bibliographystyle{plain}
\end{singlespace}



% % Lampiran
% \begin{appendix}
% 	\include{markLampiran}
% 	\setcounter{page}{2}
% 	% %-----------------------------------------------------------------------------%
\addChapter{Lampiran 1 : Kode Sumber}
\chapter*{Lampiran 1 : Kode Sumber}
%-----------------------------------------------------------------------------%
\section*{\code{admin\_useraddmaster}} \label{cha:lampir-admin}
Skrip ini diletakkan pada direktori \co{/usr/sesuatu} dan hanya dapat dieksekusi oleh \f{root}. Skrip ini berguna untuk menambahkan pengguna baru sesuai dengan konfigurasi baru yang telah ditetapkan.
\begin{lstlisting}[style=L,caption={Skrip menambahkan pengguna baru},label={lst:adduser}]
#!/bin/csh -f
blah blah blah
blah blah blah
blah blah blah
blah blah blah
blah blah blah
\end{lstlisting}

\section*{\code{getuser.cron}} \label{cha:lampir-cronadmin}
Penjelasan skrip disini
\begin{lstlisting}[style=L,caption={\f{Cronjob} menambahkan pengguna baru},label={lst:cronadduser}]
#!/bin/bash
# Change these two lines to localize to your system:
# Adapted from /usr/local/sbin/admin_useradd

cat /dev/null > $userlist
for (( i=0; i<${#listemailto[@]}; i++ ))
do
        uname=${listusername[$i]}
        mailto=${listemailto[$i]}

        echo "User $uname created, please use torqace wisely." | mail -s "Torqace user registration" $mailto
done

\end{lstlisting}

%-----------------------------------------------------------------------------%
\addChapter{Lampiran 2 : Berkas Konfigurasi}
\chapter*{Lampiran 2 : Berkas Konfigurasi}
%-----------------------------------------------------------------------------%
\section*{compute.xml}
\begin{lstlisting}[caption={Berkas \co{compute.xml}},label={lst:excomp},language=XML]
<?xml version="1.0" standalone="no"?>
<kickstart>
<description>
	Compute node XML file
</description>
</kickstart> 
\end{lstlisting}

%-----------------------------------------------------------------------------%
\addChapter{Lampiran 8 : UAT dan Kuesioner}
%-----------------------------------------------------------------------------%
\begin{landscape}
\chapter*{Lampiran 8 : UAT dan Kuesioner}
\begin{longtable}{|c|p{7cm}|p{2.5cm}|p{3.5cm}|p{3.3cm}|p{1.8cm}|}
\caption{Tabel UAT dan Kuesioner} \label{tab:uattbl}\\
\hline
No. & \multicolumn{1}{c|}{Langkah Penggunaan} & Fitur Berjalan & Tingkat Kemudahan (1-5) & Tingkat Kepuasan (1-5) & Saran / Komentar \\ 
\cline{3-5} & & Berhasil /Tidak & 1:Sangat sulit ; \hspace{100pt} 5:sangat mudah & 1 : Sangat kecewa ; 5 : sangat puas &  \\ \hline
\multicolumn{ 6}{|>{\columncolor{headertbl}}c|}{Use Case : Login} \\ \hline
1.1 & Pengguna berada pada halaman depan torqace &  &  &  &  \\ \hline
1.2 & Pengguna memasukkan username dan password pada field yang telah disediakan.Kemudian menekan tombol 'login' &  &  &  &  \\ \hline
1.3 & Apabila Sukses, maka pengguna masuk ke dalam sistem dan dihadapkan pada menu utama &  &  &  &  \\ \hline
\multicolumn{ 6}{|>{\columncolor{headertbl}}c|}{Use Case : Register} \\ \hline
2.1 & Pengguna berada pada halaman registrasi pengguna torqace &  &  &  &  \\ \hline
2.2 & Pengguna memasukkan username,password, dan email pada field yang telah disediakan. Kemudian menekan tombol 'submit' &  &  &  &  \\ \hline
2.3 & Sistem akan mengonfirmasi masukan, dan akan mengirimkan email untuk memberitahu pengguna apabila proses pendaftaran telah selesai &  &  &  &  \\ \hline
\multicolumn{ 6}{|>{\columncolor{headertbl}}c|}{Use Case : Logout} \\ \hline
3.1 & Pengguna memilih menu untuk melakukan logout &  &  &  &  \\ \hline
3.2 & Sistem akan mengeluarkan pengguna, dan pengguna tidak dapat menggunakan fitur-fitur utama aplikasi &  &  &  &  \\ \hline
\multicolumn{ 6}{|>{\columncolor{headertbl}}c|}{Use Case : Upload Job Sederhana} \\ \hline
4.1 & Pengguna memilih menu upload file/project pada menu utama &  &  &  &  \\ \hline
4.2 & Pengguna memilih pilihan 'single file' pada tipe project &  &  &  &  \\ \hline
4.3 & Pengguna memilih berkas yang akan diunggah, mengisi label, dan menentukan apakah akan menimpa project sebelumnya dengan nama yang sama atau tidak &  &  &  &  \\ \hline
4.4 & Pengguna menekan tombol 'submit' dan mengonfirmasi  &  &  &  &  \\ \hline
4.5 & Sistem akan menampilkan informasi terkait berkas yang diupload &  &  &  &  \\ \hline
\multicolumn{ 6}{|>{\columncolor{headertbl}}c|}{Use Case : Upload Job Compressed} \\ \hline
5.1 & Pengguna memilih menu upload file/project pada menu utama &  &  &  &  \\ \hline
5.2 & Pengguna memilih pilihan 'compressed files' pada tipe project &  &  &  &  \\ \hline
5.3 & Pengguna memilih arsip yang akan diunggah, mengisi label, menentukan akan melakukan make atau tidak dan menentukan apakah akan menimpa project sebelumnya dengan nama yang sama atau tidak &  &  &  &  \\ \hline
5.4 & Pengguna menekan tombol 'submit' dan mengonfirmasi  &  &  &  &  \\ \hline
5.5 & Sistem akan menampilkan informasi terkait berkas yang diupload dan diekstrak. Keluaran make juga akan ditampilkan bila dipilih &  &  &  &  \\ \hline
\multicolumn{ 6}{|>{\columncolor{headertbl}}c|}{Use Case : Upload Array Job} \\ \hline
6.1 & Pengguna memilih menu upload file/project pada menu utama &  &  &  &  \\ \hline
6.2 & Pengguna memilih pilihan 'array' pada tipe project &  &  &  &  \\ \hline
6.3 & Pengguna memilih arsip-arsip yang akan diunggah, mengisi label, menentukan akan melakukan make atau tidak dan menentukan apakah akan menimpa project sebelumnya dengan nama yang sama atau tidak &  &  &  &  \\ \hline
6.4 & Pengguna menekan tombol 'submit' dan mengonfirmasi  &  &  &  &  \\ \hline
6.5 & Sistem akan menampilkan informasi terkait berkas yang diupload dan diekstrak. Keluaran make juga akan ditampilkan bila dipilih &  &  &  &  \\ \hline
\multicolumn{ 6}{|>{\columncolor{headertbl}}c|}{Use Case : Melihat antrian pada queue} \\ \hline
7.1 & Pengguna memilih menu  queue status pada menu utama &  &  &  &  \\ \hline
7.2 & Pengguna berada pada halaman yang berisi informasi queue &  &  &  &  \\ \hline
\multicolumn{ 6}{|>{\columncolor{headertbl}}c|}{Use Case : Melihat detil antrian} \\ \hline
8.1 & Dari halaman status queue, pengguna memilih job tertentu &  &  &  &  \\ \hline
8.2 & Informasi mengenai detil job tersebut ditampilkan dalam bentuk tabel &  &  &  &  \\ \hline
8.2.1 & Apabila job tersebut bukan milik pengguna, maka sistem akan melarang pengguna melihat informasi detil suatu job &  &  &  &  \\ \hline
\multicolumn{ 6}{|>{\columncolor{headertbl}}c|}{Use Case : Membuat script job} \\ \hline
9.1 & Pengguna memilih untuk melakukan 'generate script' baik dari laporan upload berkas, atau dari penjelajahan direktori &  &  &  &  \\ \hline
9.2 & Pengguna mengisi nama job, parameter job, dan script yang akan dijalankan.  &  &  &  &  \\ \hline
9.3 & Pengguna mengonfirmasi konfirmasi submit job &  &  &  &  \\ \hline
9.4 & Pengguna dapat melihat informasi script secara keseluruhan dan pesan apakah terjadi kegagalan atau tidak, serta id job yang diberikan &  &  &  &  \\ \hline
\multicolumn{ 6}{|>{\columncolor{headertbl}}c|}{Use Case : Load spesifikasi job lain} \\ \hline
10.1 & Pengguna berada pada halaman untuk membuat script &  &  &  &  \\ \hline
10.2 & Pengguna memilih 'Load a Previous Job' &  &  &  &  \\ \hline
10.3 & Pengguna memilih job mana yang akan dimuat dan menekan tombol 'Load' &  &  &  &  \\ \hline
10.4 & Pengguna kembali ke halaman pembuatan script dengan spesifikasi job sebelumnya &  &  &  &  \\ \hline
\multicolumn{ 6}{|>{\columncolor{headertbl}}c|}{Use Case : Menjelajah Direktori} \\ \hline
11.1 & Pengguna memilih menu  'View File/Project'  pada menu utama &  &  &  &  \\ \hline
11.2 & Pengguna dapat melakukan navigasi untuk masuk ke dalam direktori tertentu, atau kembali ke direktori diatasnya, dan dapat melihat terdapat berkas apa saja dalam direktori &  &  &  &  \\ \hline
\multicolumn{ 6}{|>{\columncolor{headertbl}}c|}{Use Case : Menghapus Berkas/Direktori} \\ \hline
12.1 & Pengguna berada pada halaman penjelajahan direktori &  &  &  &  \\ \hline
12.2 & Pengguna memilih pilihan untuk menghapus berkas/direktori di samping item yang akan dihapus &  &  &  &  \\ \hline
12.3 & Pengguna mengonfirmasi konfirmasi penghapusan &  &  &  &  \\ \hline
\multicolumn{ 6}{|>{\columncolor{headertbl}}c|}{Use Case : Mengunduh Berkas/Direktori} \\ \hline
13.1 & Pengguna berada pada halaman penjelajahan direktori &  &  &  &  \\ \hline
13.2 & Pengguna memilih pilihan untuk mengunduh berkas/direktori di samping item yang akan dihapus &  &  &  &  \\ \hline
\multicolumn{ 6}{|>{\columncolor{headertbl}}c|}{Use Case : Melihat Berkas} \\ \hline
14.1 & Pengguna berada pada halaman penjelajahan direktori &  &  &  &  \\ \hline
14.2 & Pengguna memilih berkas yang berupa berkas teks &  &  &  &  \\ \hline
14.3 & Sistem akan menampilkan konten dari berkas tersebut &  &  &  &  \\ \hline
\end{longtable}
\end{landscape}
% \end{appendix}

\end{document}
