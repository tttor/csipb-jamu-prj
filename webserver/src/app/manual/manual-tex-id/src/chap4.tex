%% This is an example first chapter.  You should put chapter/appendix that you
%% write into a separate file, and add a line \include{yourfilename} to
%% main.tex, where `yourfilename.tex' is the name of the chapter/appendix file.
%% You can process specific files by typing their names in at the 
%% \files=
%% prompt when you run the file main.tex through LaTeX.
\chapter{Tinjauan Pustaka}
% Tinjauan pustaka mencakup penulisan literatur secara jelas, singkat, sistematis, dan relevan dengan
% topik penelitian. Literatur dapat berupa paparan hasil penelitian mutakhir lainnya yang sudah ada dan
% berhubungan dengan penelitian yang akan dilakukan.

Literatur yang terkait dapat dikelompokkan menjadi dua, yaitu metode prediksi dan web-server.

\section{Metode prediksi}
Berdasarkan~\cite{Chen17082015}, metode-metode prediksi yang telah dikembangkan terbagi menjadi dua kelompok besar, yaitu berbasis nerwork dan berbasis machine-learning.
Metode yang kami usulkan tergolong ke metode yang berdasarkan machine-learning.
Kami berargumen bahwa metode tersebut dapat meningkatkan akurasi prediksi drug-target, yaitu interaksi senyawa-protein.
Hal ini mungkin karena metode yang kami usulkan memanfaatkan informasi dari tiga ruang sekaligus, yaitu ruang kimia, genomik dan farmakologi.

\subsection{Metode prediksi berbasis network}
Yang~et~al.~\cite{yang} mengembangkan algoritma komputasi untuk mengambil kesimpulan target obat yang potensial dengan menganalisis secara sistematik perubahan diantara ``disease state'' dan ``desire state'' dalam jaringan penyakit. 
Metode ini sikenal dnegan MTOI atau multiple target optimal intervention. 
Sementara itu, Campillos~et~al.~\cite{campillos} mengusulkan metode dengan  menggunakan efek similarity obat untuk mengidentifikasi apakah ada interaksi antara dua obat pada target yang sama.  
Campillos mengusulkan fungsi sigmoid dimana memodelkan peluang dari berbagi target yang sama berdasarkan kesamaan informasi kimia dan fungsi linier yang memodelkan peluang berbagi target yang sama berdasarkan kesamaan efek samping yang diukur. 

Chen~et~al.~\cite{chen} membangun sebuah model berdasarkan Network-based Random Walk with Restart on the Heterogeneous network (NRWRH).
Secara garis besar, model itu menggunakan random-walk pada jaringan heterogenous untuk memprediksi pontensi interaksi senyawa-protein.
Cheng~et~al.~\cite{cheng} mengajukan tiga model prediksi supervised, yaitu 
drug-based similarity inference (DBSI), target-based similarity inference (TBSI), dan network-based inference (NBI).


\subsection{Metode prediksi berbasis machine-learning}
Saat ini, sejumlah metode pembelajaran berbasis machine learning telah dikembangkan untuk mengidentifikasi hubungan antara obat dan protein target dalam skala besar. 
Dalam tulisan ini, kami terutama memperkenalkan metode pembelajaran diawasi dan metode pembelajaran semi-diawasi. 
Dalam sebagian besar pembelajaran berbasis metode, berbagai jenis set data biologis telah terintegrasi, seperti struktur kimia obat, urutan protein target dan dikenal interaksi drug-target.

Bleakey dan Yamanishi~\cite{bleakley_yamanishi} membuat sebuah metode supervised-learning bernama Bipartite Local Model (BLM)
untuk prediksi interaksi senyawa-protein dengan mentransformasi masalah prediksi edge menjadi masalah klasifikasi biner.
Mei~et~al.~\cite{mei} menyajikan sebuah metode prediksi yang mengintegrasikan Neighbor-based Interaction-profile Inferring (NII) dengan BLM.

Wang dan Zeng~\cite{wang} mengusulkan sebuah metode yang memprediksi tidak hanya interaksi biner senyawa-protein, tetapi juga tipe interaksi.
Metode mereka berdasarkan restrited Boltzmann machines yang berdasarkan jaringan drug-target multidimensional.
Berbasiskan algoritme random forest, Cao~et~al.~\cite{cao} mengajukan metode prediksi skala besar dengan mengintegrasikan karakteristik kimia dan biologi.

\section{Web-server}
Berdasarkan observasi terhadap sejumlah web-server untuk interaksi senyawa-protein atau drug-target, kami menyimpulkan bahwa keunggulan dari web-server kami adalah adanya fasilitas untuk memasukkan jenis tanaman, sebagai obat herbal, dan jenis penyakit, sebagai target obat.
Web-server kami juga memiliki mirror database untuk mengintegrasikan informasi-informasi yang relevan sehingga akses lebih cepat.
Selain itu, kami mengusulkan web-server dengan antar muka yang intuitif serta dokumentasi kode yang lengkap.
Dengan demikian, tingkat reusability dan scalability web-server kami tinggi.

SwissTargetPrediction~\cite{HeckerAEDMEGBP12} adalah web server untuk menyimpulkan target dari molekul bioaktif berdasarkan kombinasi dari nilai similaritas ligant dalan 2D dan 3D.
SuperPred~\cite{superpred} adalah web server untuk memprediksi kode Anatomical Theraputic Chemical (ATC) dan target dari molekul.
Dinies~\cite{YamanishiKMSKG14} adalah web server untuk mengestimasi jaringan interakti senyawa-protein yang potensial.
Setiap dataset yang masuk akan diubah menjadi similaritas kernel, kemudian diproses dengan sejumlah metode machine-learning.

